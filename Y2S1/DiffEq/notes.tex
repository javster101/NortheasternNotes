\documentclass{article}
\usepackage[utf8]{inputenc}
\title{Differential Equations: MATH2341}
\author{Javier Coindreau}

\newtheorem{definition}{Definition}
\newtheorem{theorem}{Theorem}
\newtheorem{example}{Example}
\newtheorem{derivation}{Derivation}

\usepackage{amssymb}
\usepackage{enumerate}

\begin{document}

\begin{titlepage}
	\maketitle
	\tableofcontents{}
\end{titlepage}


\section{Chapter 2}

\subsection{2nd Order Differential Equations}
A 2nd order differential equation is an equation containing a 2nd order derivative of form $a_2 x y'' + a_1 x y' + a_0 x y = f(x)$, where $a_2, a_1, a_0, f(x)$ are all continuous functions.
If $f(x) \neq 0$, then it is a non-homogeneous equation, otherwise it is homogeneous.

For our purposes, $a_2, a_1, a_0$ will all be nonzero constants, allowing for easier solutions.
\subsubsection{Verifying DE \& Solutions}
How many solutions do we expect for $ay''+by'+cy=0$?


\begin{example} $y''-y'-6y=0$
	\begin{enumerate}
		\item Verify that $y_1x=e^{3x}$
			\\ Plug in: $y_1=e^{3x}, y_2=3e{3x}, y_3=9e{3x}$
			\\ $9e^{3x}-3e^{3x}-6e^{3x}=0$
			\\ Valid solution
		\item Verify that $y_1=2e^{3x}$
			\\ Plug in $y_1=2e^{3x}, y_2=6e{3x}, y_3=18e{3x}$
			\\ $18e^{3x}-6e^{3x}-12e^{3x}=0$
			\\ Valid solution
		\item Verify that $y_1=2e^{-2x}$
			\\Plug in $y_1=e^{-2x}, y_2=-2e{-2x}, y_3=4e{-2x}$
			\\ $4e^{3x}+2e^{3x}-6e^{3x}=0$
			\\ Valid solution
	\end{enumerate}
\end{example}

Any second order differential equation has at most two linearly independent solutions.
\begin{definition}
	Two functions $y_1$ and $y_2$ are linearly independent $\iff y_1 \neq ky_2, k \in \mathbb{R}$
\end{definition}

\subsubsection{Solutions \& Superposition}
Suppose that $y_1, y_2$ are two linearly independent functions that satisfy a linear homogeneous differential equation. For such solutions, a general solution to the differential equation is 
\begin{equation} y(x)=C_1 y_1(x)+C_2 y_2(x); C_1,C_2 \in \mathbb{R} \end{equation}

\begin{itemize}
	\item $e^{r_1 x},e^{r_2 x}$ are independent if $r_1 \neq r_2 $
	\item $\sin{px}, \sin{qx}$ are independent if $p \neq q$
\end{itemize}

\begin{example} $y''-y'-6y=0$
	\\ In this case, the general solution is $y(x)=C_1e^{3x}+C_2e^{-2x}$.
\end{example}

\subsection{2.3: Solving 2nd order homogeneous equations with constant coefficients}

	What function has a $y, y', y''$ that are just scalar multiples of $y$?
	
	$e^{rx}$ always works, as $y=e^{rx}, y'=re^{rx}, y''=r^2e^{rx}$. When plugged in, this results in $ar^2e^{rx}+bre^{rx}+ce^{rx}=0$. $e^{rx}(ar^2+br+c)=0$. As the left term never becomes 0, the right term must.
	
Solve: $ar^2+br+c=0$
\begin{indent}
\\	This is known as the characteristic equation, as it resembles the original.
\end{indent}
\\Solutions can be derived from the quadratic function: $r_1, r_2=\frac{-b +- \sqrt{b^2-4ac}}{2a}$
	\\From this, we find $r_1, r_2$ and get the two solutions, $e^{r_1 x}, e^{r_2 x}$
\\3 possiblities arise from this.
\begin{enumerate}
	\item $b^2>4ac$.
		\\$r_1=\frac{-b + \sqrt{b^2-4ac}}{2a}, r_2=\frac{-b -\sqrt{b^2-4ac}}{2a}$
	\item $b^2=4ac$ (Be careful!)
		\\$r_1=r_2=\frac{-b}{2a}$. This cannot happen, both are linearly dependent and we need two linearly independent solution! If $y_1=e^{r_1 x}$, how can we get a linearly independent function $y^2$ that satisfies the same equation?
		\\Trick is to multiply $y_1$ with an extra $x$, resulting in $xe^{r_1 x}$. This works due to Abel's Theorem.
		\begin{theorem}
		If $y_1(x)$ is a solution to the equation $y''+p(x)y'+q(x)y=0$, then the second linearly independent solution is given by $y_2(x)=y_1(x)*\int{\frac{e^{-\int{p(x)dx}}}{(y_1(x))^2}dx}.$\end{theorem}
		Our equation, in standard form, becomes $y''+\frac{b}{a}y'+\frac{c}{a}y'=0$.
	$p(x)$ becomes $\frac{b}{a}$, and evaluating the integral results in $xe^{r_1}x$.
	\\The general solution results in $y(x)=C_1 e^{r_1 x}+C_2 xe^{r_1 x}$
	\item $b^2<4ac$
		\\$\frac{-b + \sqrt{b^2-4ac}}{2a}=\frac{-b \pm i\sqrt{b^2-4ac}}{2a}, 
		\frac{-b}{2a}\pm i \frac{\sqrt{4ac-b^2}}{2a}$.
		\\We want to remove $i$. 
		To do so, we simplify this to $p+iq$, where $p,q \in \mathbb{R}$. 
		We want to find two real valued independent functions.
		
		$e^{r_1 x}=e^{(p+iq)x}=e^{px}e^{iqx}, e^{r_2x}=e^{(p-iq)x}=e^{px}e^{-iqx}$.
		To solve this, we use Euler's Identity:
		\begin{derivation}
			\begin{equation}
				e^{i\theta}=\cos{\theta}+i \sin{\theta}
			\end{equation}
			$e^\theta=1+\theta+\frac{\theta^2}{2!}..., 
			\cos{\theta}=1-\frac{\theta^2}{2!}...,
			\sin{\theta}=\theta-..., 
			\\e^{i\theta}=\cos{\theta}+i \sin{\theta}$ 
			due to presence of imaginary constant.
		\end{derivation}
		We can use this to extract the $i$ from the exponent: 
		$e^{px}e^{iqx}=e^{px}(\cos{px}+i\sin{qx}, e^{px}e^{iqx}=e^{px}(\cos{px}-i\sin{qx}, 
		\\y_1(x)=e^{px}\cos{qx}, y_2(x)=e^{px}\sin{qx}$.
		
		The general solution becomes 
		\begin{equation}
			y(x)=C_1e^{px}\cos{qx}+C_2e^{px}\sin{qx}=e^{px}(C_1\cos{qx}+C_2\sin{qx}
		\end{equation}
		Note: if $p=0, y(x)=C_1\cos{qx}+C_2\sin{qx}$ and becomes an oscillatory motion. 
		If $p>0$, oscillations get stronger over time, and if $p<0$ motion becomes dampened.
\end{enumerate}

\begin{example} $y''-y'-6y=0$. \\The characteristic equation is $r^2-r-6=0$, which is factorable here to $(r-3)(r+2)=0, r_1=3, r_2=-2$. From this, the general solution becomes $y(x)=C_1e^{3x}+C_2e^{-2x}$.
\end{example}
\begin{example} $y''+6y'+9y=0$
	\\The characteristic equation is $r^2+6r+9=0$ $6^2=4*1*9, 36=36$, root at $-3$. 
	The general solution becomes $y(x)=C_1 e^{-3x}+C_2xe^{-3x}$
\end{example}
\begin{example}
	$y''+16y=0$
	\\The characteristic equation is $r^2+16=0, r^2=-16, r=\pm\sqrt{-16}=0\pm4i$.
	The general solition becomes $y(x)=C_1\cos{4x}+C_2\sin{4x}$
\end{example}
\begin{example}
	Find a particular solution for 
	\begin{enumerate}
		\item $y''+4y'+29y=0$
			\\Not factorable, quadratic equation results in
			$\frac{-4}{2} \pm \frac{10i}{2}$
			\\$p=-2, q=5i, y(x)=e^{-2x}[C_1 \cos{5x} + C_2 \sin{5x}]$
			\\ Solve when $y(0)=1 
			\\y'(0)=1$: $y(0)=e^1 [C_1 \cos{0} + C_2 \sin{0} = 1*(C_1*1), C_1=1$
			\\ Solve when $y'(0)=1$
			\\$y'(x)=-2e^{-2x}[C_1 \cos{5x} + C_2 \sin{5x}]+(-5 C_1 \sin{5x}+5 C_2 \cos{5x})$
			\\$y'(0)=-2[C_1*1+C_2*0]+(-5 C_1 * 0 + 5 C_2 * 1) = -2C_1+5C_2, C_2=\frac{2}{5}$
			\\The particular solution becomes $y_p(x)=e^{-2x}[\cos{5x}+\frac{2}{5}\sin{5x}]$
		\item $y''+3y'+2y=0$
			\\Factors to $(r+2)(r+1), r_1=-2, r_2=-1, y(x)=C_1e^{-2x}+C_2e^{-1x}$
			\\Solve for $y(0)=1$
			\\$y(0)=C_1+C_2=1$
			\\Solve for $y'(0)=1$
			\\$y'(x)=-2 C_1 e^{-2x}-C_2e^{-1x}=-2 C_1- C_2=1$
			\\Solving for $C_1$ gives $C_1=-2, C_2=3$
			\\The particular solution becomes $y(x)=-2e^{-2x}+3e^{-x}$
		\item $9''+6y'+9y=0$
			\\Factors to one root at $-3, y(x)=C_1e^{-3x}+C_2xe^{-3x}$
			\\Solve for $y(0)=2$: $y(0)=C_1=2$
			\\Solve for $y'(0)=3$
			\\$y'(x)=-3C_1e^{-3x}+C_2e^{-3x}-3C_2e^{-3x}$
			\\$y'(0)==3C_1+C_2-3C_2=3*2-2C_2, C_2=9$
			\\Particular solution becomes $y(x)=2e^{-3x}+yxe^{-3x}$
	\end{enumerate}
\end{example}
\subsection{2.4: Applications}
\subsubsection{Free Mechanical Vibrations}
Given a spring attached to a mass with a dampener, what mathematical model can simulate its motion? Since it is a free vibration, it has no external forces. This means that it is guided purely by the force for the spring and the dampener.
Since we are observing accelerated motion, it will be a 2nd order differential equation ($a=\frac{d^2x}{dt^2}$).
The total force from this spring is given by Hooke's law, 
\begin{equation} F_s = -kx, k=\frac{|F_s|}{m} > 0 \end{equation}
The damping force exerted on the system by the dashpot is proportional to its velocity,
\begin{equation} F_d = -cv \end{equation}
where $c$ is the damping constant.

We know that the velocity $v=\frac{dx}{dt}$, so the total force becomes $F=F_s+F_d$, and replacing fixed values with differentials becomes $m \frac{d^2x}{dt^2} = -kx -c \frac{dx}{dt}$, or alternatively $mx''+cx'+kx=0$.

Being a real world problem, we know that $m$ must be greater than 0, $c$ may be 0 if there is no dashpot, and $k$ is greater than 0 due to the presence of the spring.
At equilibrium, $x(0)=0, x'(0)=0$ as the object starts at rest. In this case, we can take compression as movement in the negative $x$ axis and stretching in the positive $x$ axis.

Now, let's go through different cases.
\begin{enumerate}[a: ]
	\item No dashpot (undamped oscillation, $c=0$):
		\\Since it is undampened, this oscillation should never end. 
		The equation in this case would be $mx''+kx=0$, simplifying to $x''+\frac{k}{m}x=0$. 
		We can then extract the constant $\omega$, the natural frequency of the system, as $\omega=\sqrt{\frac{k}{m}}, \omega^2=\frac{k}{m}$.
		\\The characteristic equation here becomes 
		$r^2+\omega^2=0, r^2=-\omega^2, r=\pm i\omega$.
		This equation will consist of purely oscillatory terms, and will become 
		$x(t)=C_1 \cos{\omega t}+C_2 \sin{\omega t}$.
		It can be confusing for it to have both sin and cos, so we can rewrite this in amplitude-phase form, resulting in $x(t)=A \cos{(\omega t-\phi)}$, where $\phi=\tan^{-1}{C_2/C_1}$.
	\begin{derivation}
		\begin{equation} x(t)=A \cos{(\omega t-\phi)} \end{equation}
		From original, we create $\sqrt{C_1^2+C_2^2}(\frac{C_1 \cos{\omega t}}{\sqrt{C_1^2+C_2^2}}+\frac{C_1 \sin{\omega t}}{\sqrt{C_1^2+C_2^2}})$. 
		We then extract $A$, which becomes the length of a triangle with sides $C_1,C_2$, from the Pythagorean Theorem, creating a new function $A(\frac{\cos {\omega t}}{A} + \frac{\sin {\omega t}}{A})$. We can then use the trig identity $\cos{(A-B)}=\cos{A}\cos{B}+\sin{A}\sin{B}$ to simplify and create the final function.
	\end{derivation}
\end{enumerate}
\end{document}
