\documentclass{article}
\usepackage[utf8]{inputenc}
\title{Differential Equations: MATH2341}
\author{Javier Coindreau}

\newtheorem{definition}{Definition}
\newtheorem{theorem}{Theorem}
\newtheorem{example}{Example}
\newtheorem{derivation}{Derivation}

\usepackage{amssymb}
\usepackage{enumerate}
\usepackage{mathtools}
\usepackage{mathrsfs,relsize,array}

\newcommand\laplace{\mathlarger{\mathlarger{\mathscr{L}}}}

\begin{document}

\begin{titlepage}
	\maketitle
	\tableofcontents{}
\end{titlepage}


\section{Chapter 2}

\subsection{2nd Order Differential Equations}
A 2nd order differential equation is an equation containing a 2nd order derivative of form $a_2 x y'' + a_1 x y' + a_0 x y = f(x)$, where $a_2, a_1, a_0, f(x)$ are all continuous functions.
If $f(x) \neq 0$, then it is a non-homogeneous equation, otherwise it is homogeneous.

For our purposes, $a_2, a_1, a_0$ will all be nonzero constants, allowing for easier solutions.
\subsubsection{Verifying DE \& Solutions}
How many solutions do we expect for $ay''+by'+cy=0$?


\begin{example} $y''-y'-6y=0$
	\begin{enumerate}
		\item Verify that $y_1x=e^{3x}$
			\\ Plug in: $y_1=e^{3x}, y_2=3e{3x}, y_3=9e{3x}$
			\\ $9e^{3x}-3e^{3x}-6e^{3x}=0$
			\\ Valid solution
		\item Verify that $y_1=2e^{3x}$
			\\ Plug in $y_1=2e^{3x}, y_2=6e{3x}, y_3=18e{3x}$
			\\ $18e^{3x}-6e^{3x}-12e^{3x}=0$
			\\ Valid solution
		\item Verify that $y_1=2e^{-2x}$
			\\Plug in $y_1=e^{-2x}, y_2=-2e{-2x}, y_3=4e{-2x}$
			\\ $4e^{3x}+2e^{3x}-6e^{3x}=0$
			\\ Valid solution
	\end{enumerate}
\end{example}

Any second order differential equation has at most two linearly independent solutions.
\begin{definition}
	Two functions $y_1$ and $y_2$ are linearly independent $\iff y_1 \neq ky_2, k \in \mathbb{R}$
\end{definition}

\subsubsection{Solutions \& Superposition}
Suppose that $y_1, y_2$ are two linearly independent functions that satisfy a linear homogeneous differential equation. For such solutions, a general solution to the differential equation is 
\begin{equation} y(x)=C_1 y_1(x)+C_2 y_2(x); C_1,C_2 \in \mathbb{R} \end{equation}

\begin{itemize}
	\item $e^{r_1 x},e^{r_2 x}$ are independent if $r_1 \neq r_2 $
	\item $\sin{px}, \sin{qx}$ are independent if $p \neq q$
\end{itemize}

\begin{example} $y''-y'-6y=0$
	\\ In this case, the general solution is $y(x)=C_1e^{3x}+C_2e^{-2x}$.
\end{example}

\subsection{2.3: Solving 2nd order homogeneous equations with constant coefficients}

	What function has a $y, y', y''$ that are just scalar multiples of $y$?
	
	$e^{rx}$ always works, as $y=e^{rx}, y'=re^{rx}, y''=r^2e^{rx}$. When plugged in, this results in $ar^2e^{rx}+bre^{rx}+ce^{rx}=0$. $e^{rx}(ar^2+br+c)=0$. As the left term never becomes 0, the right term must.
	
Solve: $ar^2+br+c=0$
\begin{indent}
\\	This is known as the characteristic equation, as it resembles the original.
\end{indent}
\\Solutions can be derived from the quadratic function: $r_1, r_2=\frac{-b +- \sqrt{b^2-4ac}}{2a}$
	\\From this, we find $r_1, r_2$ and get the two solutions, $e^{r_1 x}, e^{r_2 x}$
\\3 possiblities arise from this.
\begin{enumerate}
	\item $b^2>4ac$.
		\\$r_1=\frac{-b + \sqrt{b^2-4ac}}{2a}, r_2=\frac{-b -\sqrt{b^2-4ac}}{2a}$
	\item $b^2=4ac$ (Be careful!)
		\\$r_1=r_2=\frac{-b}{2a}$. This cannot happen, both are linearly dependent and we need two linearly independent solution! If $y_1=e^{r_1 x}$, how can we get a linearly independent function $y^2$ that satisfies the same equation?
		\\Trick is to multiply $y_1$ with an extra $x$, resulting in $xe^{r_1 x}$. This works due to Abel's Theorem.
		\begin{theorem}
		If $y_1(x)$ is a solution to the equation $y''+p(x)y'+q(x)y=0$, then the second linearly independent solution is given by $y_2(x)=y_1(x)*\int{\frac{e^{-\int{p(x)dx}}}{(y_1(x))^2}dx}.$\end{theorem}
		Our equation, in standard form, becomes $y''+\frac{b}{a}y'+\frac{c}{a}y'=0$.
	$p(x)$ becomes $\frac{b}{a}$, and evaluating the integral results in $xe^{r_1}x$.
	\\The general solution results in $y(x)=C_1 e^{r_1 x}+C_2 xe^{r_1 x}$
	\item $b^2<4ac$
		\\$\frac{-b + \sqrt{b^2-4ac}}{2a}=\frac{-b \pm i\sqrt{b^2-4ac}}{2a}, 
		\frac{-b}{2a}\pm i \frac{\sqrt{4ac-b^2}}{2a}$.
		\\We want to remove $i$. 
		To do so, we simplify this to $p+iq$, where $p,q \in \mathbb{R}$. 
		We want to find two real valued independent functions.
		
		$e^{r_1 x}=e^{(p+iq)x}=e^{px}e^{iqx}, e^{r_2x}=e^{(p-iq)x}=e^{px}e^{-iqx}$.
		To solve this, we use Euler's Identity:
		\begin{derivation}
			\begin{equation}
				e^{i\theta}=\cos{\theta}+i \sin{\theta}
			\end{equation}
			$e^\theta=1+\theta+\frac{\theta^2}{2!}..., 
			\cos{\theta}=1-\frac{\theta^2}{2!}...,
			\sin{\theta}=\theta-..., 
			\\e^{i\theta}=\cos{\theta}+i \sin{\theta}$ 
			due to presence of imaginary constant.
		\end{derivation}
		We can use this to extract the $i$ from the exponent: 
		$e^{px}e^{iqx}=e^{px}(\cos{px}+i\sin{qx}, e^{px}e^{iqx}=e^{px}(\cos{px}-i\sin{qx}, 
		\\y_1(x)=e^{px}\cos{qx}, y_2(x)=e^{px}\sin{qx}$.
		
		The general solution becomes 
		\begin{equation}
			y(x)=C_1e^{px}\cos{qx}+C_2e^{px}\sin{qx}=e^{px}(C_1\cos{qx}+C_2\sin{qx}
		\end{equation}
		Note: if $p=0, y(x)=C_1\cos{qx}+C_2\sin{qx}$ and becomes an oscillatory motion. 
		If $p>0$, oscillations get stronger over time, and if $p<0$ motion becomes dampened.
\end{enumerate}

\begin{example} $y''-y'-6y=0$. \\The characteristic equation is $r^2-r-6=0$, which is factorable here to $(r-3)(r+2)=0, r_1=3, r_2=-2$. From this, the general solution becomes $y(x)=C_1e^{3x}+C_2e^{-2x}$.
\end{example}
\begin{example} $y''+6y'+9y=0$
	\\The characteristic equation is $r^2+6r+9=0$ $6^2=4*1*9, 36=36$, root at $-3$. 
	The general solution becomes $y(x)=C_1 e^{-3x}+C_2xe^{-3x}$
\end{example}
\begin{example}
	$y''+16y=0$
	\\The characteristic equation is $r^2+16=0, r^2=-16, r=\pm\sqrt{-16}=0\pm4i$.
	The general solition becomes $y(x)=C_1\cos{4x}+C_2\sin{4x}$
\end{example}
\begin{example}
	Find a particular solution for 
	\begin{enumerate}
		\item $y''+4y'+29y=0$
			\\Not factorable, quadratic equation results in
			$\frac{-4}{2} \pm \frac{10i}{2}$
			\\$p=-2, q=5i, y(x)=e^{-2x}[C_1 \cos{5x} + C_2 \sin{5x}]$
			\\ Solve when $y(0)=1 
			\\y'(0)=1$: $y(0)=e^1 [C_1 \cos{0} + C_2 \sin{0} = 1*(C_1*1), C_1=1$
			\\ Solve when $y'(0)=1$
			\\$y'(x)=-2e^{-2x}[C_1 \cos{5x} + C_2 \sin{5x}]+(-5 C_1 \sin{5x}+5 C_2 \cos{5x})$
			\\$y'(0)=-2[C_1*1+C_2*0]+(-5 C_1 * 0 + 5 C_2 * 1) = -2C_1+5C_2, C_2=\frac{2}{5}$
			\\The particular solution becomes $y_p(x)=e^{-2x}[\cos{5x}+\frac{2}{5}\sin{5x}]$
		\item $y''+3y'+2y=0$
			\\Factors to $(r+2)(r+1), r_1=-2, r_2=-1, y(x)=C_1e^{-2x}+C_2e^{-1x}$
			\\Solve for $y(0)=1$
			\\$y(0)=C_1+C_2=1$
			\\Solve for $y'(0)=1$
			\\$y'(x)=-2 C_1 e^{-2x}-C_2e^{-1x}=-2 C_1- C_2=1$
			\\Solving for $C_1$ gives $C_1=-2, C_2=3$
			\\The particular solution becomes $y(x)=-2e^{-2x}+3e^{-x}$
		\item $9''+6y'+9y=0$
			\\Factors to one root at $-3, y(x)=C_1e^{-3x}+C_2xe^{-3x}$
			\\Solve for $y(0)=2$: $y(0)=C_1=2$
			\\Solve for $y'(0)=3$
			\\$y'(x)=-3C_1e^{-3x}+C_2e^{-3x}-3C_2e^{-3x}$
			\\$y'(0)==3C_1+C_2-3C_2=3*2-2C_2, C_2=9$
			\\Particular solution becomes $y(x)=2e^{-3x}+yxe^{-3x}$
	\end{enumerate}
\end{example}
\subsection{2.4: Applications}
\subsubsection{Free Mechanical Vibrations}
Given a spring attached to a mass with a dampener, what mathematical model can simulate its motion? Since it is a free vibration, it has no external forces. This means that it is guided purely by the force for the spring and the dampener.
Since we are observing accelerated motion, it will be a 2nd order differential equation ($a=\frac{d^2x}{dt^2}$).
The total force from this spring is given by Hooke's law, 
\begin{equation} F_s = -kx, k=\frac{|F_s|}{m} > 0 \end{equation}
The damping force exerted on the system by the dashpot is proportional to its velocity,
\begin{equation} F_d = -cv \end{equation}
where $c$ is the damping constant.

We know that the velocity $v=\frac{dx}{dt}$, so the total force becomes $F=F_s+F_d$, and replacing fixed values with differentials becomes $m \frac{d^2x}{dt^2} = -kx -c \frac{dx}{dt}$, or alternatively $mx''+cx'+kx=0$.

Being a real world problem, we know that $m$ must be greater than 0, $c$ may be 0 if there is no dashpot, and $k$ is greater than 0 due to the presence of the spring.
At equilibrium, $x(0)=0, x'(0)=0$ as the object starts at rest. In this case, we can take compression as movement in the negative $x$ axis and stretching in the positive $x$ axis.

Now, let's go through different cases.
\begin{enumerate}[Opt. a: ]
	\item No dashpot (undamped oscillation, $c=0$):
		\\Since it is undampened, this oscillation should never end. 
		The equation in this case would be $mx''+kx=0$, simplifying to $x''+\frac{k}{m}x=0$. 
		We can then extract the constant $\omega$, the natural frequency of the system, as $\omega=\sqrt{\frac{k}{m}}, \omega^2=\frac{k}{m}$.
		\\The characteristic equation here becomes 
		$r^2+\omega^2=0, r^2=-\omega^2, r=\pm i\omega$.
		This equation will consist of purely oscillatory terms, and will become 
		$x(t)=C_1 \cos{\omega t}+C_2 \sin{\omega t}$.
		It can be confusing for it to have both sin and cos, so we can rewrite this in amplitude-phase form, resulting in $x(t)=A \cos{(\omega t-\phi)}$, where $\phi=\tan^{-1}{C_2/C_1}$.
	\begin{derivation}
		\begin{equation} x(t)=A \cos{(\omega t-\phi)} \end{equation}
		From original, we create $\sqrt{C_1^2+C_2^2}(\frac{C_1 \cos{\omega t}}{\sqrt{C_1^2+C_2^2}}+\frac{C_1 \sin{\omega t}}{\sqrt{C_1^2+C_2^2}})$. 
		We then extract $A$, which becomes the length of a triangle with sides $C_1,C_2$, from the Pythagorean Theorem, creating a new function $A(\frac{\cos {\omega t}}{A} + \frac{\sin {\omega t}}{A})$. We can then use the trig identity $\cos{(A-B)}=\cos{A}\cos{B}+\sin{A}\sin{B}$ to simplify and create the final function.
	\end{derivation}
	\item Damped oscillation ($c \neq 0$)
		\\The equation is $mx''+cx'+kx=0$, with a characteristic equation of $mr^2+cr+k=0$. 
		We can use the quadratic formula here to get the roots as $\frac{-c \pm \sqrt{c^2-4mk}}{2m}$,
		resulting in roots of $r_1=\frac{-c}{2m}+\frac{\sqrt{c^2-4mk}}{2m}, r_2=\frac{-c}{2m}-\frac{\sqrt{c^2-4mk}}{2m}$. 
				Since they are both negative due to the negative poking outside, we can guarantee that $r_2$ is negative since $m,k,c>0$.
				$r_1$ is also negative, since $\sqrt{x^2-b}$ will always be less than just $x$ if $b$ is positive, which it is in our case. We can then use the cases from the first section.
		\begin{enumerate}
			\item $C^2>4mk$ is considered overdampened, as there will be two real negative roots. This results in $x(t)=C_1e{r_1t}+C_2e^{r_2t}$
			\item $C^2=4mk$ results in a critically damped solution with one real root. This results in $x(t)=C_1e^{r_1t}+C_2xe^{r_1t}=e^{r_1t}(C_1+C_2x)$
			\item $C^2<4mk$ is underdampened, so oscillations will appear. 
				In this case we get $x(t)=e^{\frac{-C}{2m}}[C_1\cos{(\frac{\sqrt{4mk-C^2}}{2m})}+C_2\sin{(\frac{\sqrt{4mk-C^2}}{2m})}$. 
				This function has no natural frequency, and $q$ in this case is considered a pseudofrequency due to damping.
			This can be rewritten using amplitude-phase as $x(t)=Ae^{\frac{-C}{2m}}\cos{(\omega t-\phi)}$
		\end{enumerate}
\end{enumerate}

\subsection{Non homogeneous 2nd order differential equations}
These are of the form $ay''+by'+cy=f(x)$, where $f(x)$ is a differentiable and continuous function.
\begin{theorem}
	Given a differential equation of the form $ay''+by'+cy=f(x)$, the general solution is given by 
	\begin{equation} y(x)=y_h(x)+y_p(x) \end{equation} where $y_h(x)$ is the homogeneous solution to the
	differential equation and $y_p(x)$ is a particular solution for the non-homogeneous equation.
\end{theorem}

\subsubsection{Finding $y_p(x)$ with undetermined coefficients}

To find $y_p(x)$, we first make a guess for $y_p$ based on $f(x)$ with an unknown coefficient. This is then plugged into the differential equation  $ay_p''+by_p'+cy_p=f(x)$ and solved for the unknown coefficient.

The coefficient is chosed based on the behavior of $f(x)$

\begin{enumerate}
	\item $f(x)=\alpha e^{kx}$, the guess should be $y_p=Ae^{kx}$
	\item $f(x)=\sin{kx}$ or $\cos{kx}$. This is a difficult case, because unlike the above, the canceling step in the examples does not automatically happen.
		Therefore, we use $y_p=A\cos{(kx)}+B\sin{(kx)}$. It is very important to write both, even if only one is actually there.
	\item $f(x)=$polynomial of degree $n$, guess should be a polynomial of the same degree.
		Being a general polynomial, all terms must be written even if some terms are missing. Examples are $f(x)=2x, y_p=Ax+B$
\end{enumerate}

\begin{example}
	$y''+3y'+2y=2e^{4x}$, find the general solution.
	
	 \begin{enumerate}[Step 1:]
		 \item Always find the homogeneous solution: $y''+3y'+2y=0, r^2+3r+2=0, (r-1)(r-2)=0, r_1=1, r_2=2, y_h(x)=C_1e^x+C_2e^{2x}$
	
		 \item Find the specific solution: Guess $y_p=Ae^{4x}$. 
	\\Then, find $y', y''; y'=4Ae^{4x}, y''=16Ae^{4x}$. 
	\\Plugged in, $16Ae^{4x}-12Ae^{4x}+2Ae^{4x}=2e^{4x}$. 
	\\We can cancel $e^{4x}$, resulting in $16A-12A+2A=2, 6A-2, A=1/3$. 
	\\Therefore, $y_p=\frac{1}{3}e^{4x}$

		\item Combine both to create $y(x)=C_1e^x+C_2e^{2x}+\frac{1}{3}e^{4x}k$
	\end{enumerate}
\end{example}

\begin{example}
	$y''+3y'+2y=2sin{3x}$
	\begin{enumerate}[Step 1:]
		\item Homogeneous solution is $y_h(x)=C_1e^x+C_2e^{2x}$, same as before.

		\item Guess: $y_p=A\cos{3x}+B\sin{3x}$.
	\\$y', y'': y'=-3A\sin{3x}+3B\cos{3x}, y''=-9A\cos{3x}-9B\sin{3x}$
	\\Plugged in, 
	\\$-9A\cos{3x}-9B\sin{3x}
	\\-9B\cos{3x}+9A\sin{3x}
	\\+2A\cos{3x}+2B\sin{3x}
	\\=0\cos{3x}+2\sin{3x}$
	\\After simplifying, $(-7A-9B)\cos{3x}+(9A-7B)\cos{3x}=0\cos{3x}+2\sin{3x}$.
	Terms should match, so $9A-7B=2, -7A+9B=0, B=\frac{-7}{65}, A=\frac{9}{64}$

		\item Merge, $y(x)=C_1e^x+C_2e^{2x}+\frac{9}{65}\cos{3x}-\frac{7}{65}\sin{3x}$
	\end{enumerate}
\end{example}

\begin{example}
	$y''+3y'+2y=2x+1$
	\begin{enumerate}[Step 1:]
		\item $y_h(x)=C_1e^x+C_2e^{2x}$

		\item Guess: $y_p=Ax+B, y_p'=A, y_p''=0$
			\\Plugged in, $-3A+2Ax+2B=2x+1, A=1, B=2, y_p=x+2$
		\item Merge, $y(x)=C_1e^x+C_2e^{2x}+x+2$
	\end{enumerate}
\end{example}

What if multiple of these were combined into the same equation? What if you had $y''-3y'+2y=2e^{4x}+2\sin{3x}+2x+1$?

Fortunately, you can solve them individually and sum them together. With the previous one, for example, the result would be $y(x)=C_1e^{2x}+C_2e^{2x}-1/3 e^{4x}-\frac{7}{65}\cos{3x}+\frac{9}{65}\sin{3x}+x+2$.

Another option would be to start with a single guess containing all the equation types given, solving $y_p = Ae^{4x}+B\cos{3x}+C\sin{3x}+Dx+E$

\begin{example}
	Find the general solution for $y''-y'-2y=3e^{2x}$.

	Characteristic eq: $r^2-r-2=0, (r-2)(r+1)=0, y_h(x)=C_1e^{2x}+C_2e^{-x}$

	Guess: $y_p=Ae^{2x}, y_p'=2Ae^{2x}, y_p''=4Ae^{2x}$. 

	Eq: $4Ae^{2x}-2Ae^{2x}-2Ae^{2x}=3e^{2x}, 0=3e^{2x}$. Impossible? Options are $0=3, e^{2x}=0$. Neither are possible! Therefore, guess is wrong here.
\end{example}

Why is this not possible? Because our guess is linearly dependent with $C_1e^{2x}$, which is not allowed! Since $C_1e^{2x}$ is unchangeable, we must change the guess in the same way we'd do it before, by making the guess $Axe^{2x}$. Unfortunately, solving this brings in the product rule: $y_p'=Ae^{2x}+2Axe^{2x}, y_p''=2Ae^{2x}+2Ae^{2x}+4Axe^{2x}=4Ae^{2x}+4Axe^{2x}$. This is then used as before.

\begin{example}[10]
	$4Ae^{2x}+4Axe^{2x}-Ae^{2x}-2Axe^{2x}-2Axe^{2x}=3e^{2x}, 4Ae^{2x}-Ae^{2x}=3e^{2x}$ (Note: the terms with x must cancel, otherwise math was done wrong)

	$3Ae^{2x}=3e^{2x}, A=1, y(x)=C_1e^{2x}+C_2e^{-x}+xe^{2x}$ (Don't forget the x!)
\end{example}

Using this, we can make a better order of approach:

\begin{enumerate}
	\item Find $y_h$
	\item Write the normal $y_p$ guess
	\item Compare each term in $y_p$ to each term in $y_h$
	\item If any are linearly dependent, multiply ONLY the linearly dependent terms in $y_p$ by $x^n$ where $n$ is the smallest positive integer that ensures that the term is linearly independent.
	\item Solve for $y_p$
\end{enumerate}

\begin{example}
	Find the general solution for $y''-4y'+4y=3e^{2x}$

	$y_h$: $r^2-4r+4=0, (r-2)(r-2), y_h(x)=C_1e^{2x}+C_2xe^{2x}$
	
	$y_p$: $Ae^{2x}$: LINEARLY DEPENDENT. Lowest power of $x$ needed here is $x^2, y_p=Ax^2e^{2x}$

	Differentiate: $y_p'=2Ax^2e^{2x}+2Axe^{2x}, y''=4Axe^{2x}+4Ax^2e^{2x}+4Axe^{2x}+2Ae^{2x}$

	$y_p(x)=4Ax^2e^{2x}+8Axe^{2x}+2Ae^{2x}-8Ax^2e^{2x}-8Axe^{2x}+4Ax^2e^{2x}=2Ae^{2x}=3e^{2x}, 2A=3, A=1.5$

	$y(x)=C_1e^{2x}+C_2xe^{2x}+1.5x^2e^{2x}$
\end{example}

\subsection{Resonance}
What is an example of a resonating system?

\begin{itemize}
	\item Musical Instruments
	\item Noise cancelling headphones
	\item Kids on a swing
	\item Microwave ovens with water
\end{itemize}

In these systems, there is a natural frequency, a frequency at which the system will keep oscillating even when forced to go further.
For example, in a dashpot system $mx''+kx=F_0 \cos{\alpha t}$, resonance happens when the desired frequency $\alpha$ is equal to the ideal frequency $\sqrt{\frac{k}{m}}$

\section{Chapter 3: Laplace Transforms}
\subsection{Background}

Frequency is the inverse of time, in cycles/unit of time. We can take a differential equation in the time domain and transform it into an
algebraic equation in the frequency domain. This makes them notably easier to manipulate when compared to differentials, but
requires us to solve it in this new frequency domain. Once done, we can take it back to the time domain.

\begin{definition}
	Lef $f(t)$ be a function which is not exponentially growing. We define the laplace transform of $f(t)=\laplace[f(t)]=\int_0^\infty{e^{-st}f(t)dt}=F(s)$
\end{definition}

Why does this work?

\begin{itemize}
	\item he negative in the exponential is purely for convergence reasons.
	\item All applications of exponential growth/decay or oscillatory function are analyzable through a laplace transform, partially due
		to Euler's identity $e^{it}=\cos{t}+i\sin{dt}$
\end{itemize}

In Laplace transforms, $e^{st}=e^{at}+e^{ibt}, s=a+ib$. Using this, exponential/sinusoidal/both types of problems are analyzable.

\begin{example}
	$f(t)=k, \laplace[f(t)]=\int_0^\infty{e^{-st} k dt}=k \lim_{b\rightarrow\infty}{\int_0^\infty{e^{-st}dt}}=\frac{k}{s}$
\end{example}

\begin{example}
	$Find \laplace[e^{at}], \int_0^\infty{e^{-st}e^{at}dt}=\lim_{b\rightarrow\infty}{\int_0^b{e^{-(s-a)t}dt}}
	\\=\frac{1}{-s+a}e^{-s+a}, 0\rightarrow \infty, \frac{1}{-s+a}e^{-\infty+a}-1=\frac{1}{s-a}$

\end{example}

Doing this for every Laplace sucks, but luckily all Laplace transforms are tabulated (Available in Canvas->Modules->Handouts(?)).


\begin{example}
	Using tables:

	\begin{itemize}
		\item $\laplace[\sin{3t}]=\frac{3}{s^2+9}$
		\item $\laplace[\cos{2t}]=\frac{s}{s^2+4}$
		\item $\laplace[e^{-3t}\cos{5t}]=\frac{s+2}{(s+2)^2+25}$
	\end{itemize}
\end{example}

Generalizing, $\laplace[a f(t) \pm b g(t)]=a F(s)\pm b G(s)]$

\begin{example}
	$\laplace[2-6e^{3t}+t^4-5\cos{t}]=\frac{2}{s}-\frac{6}{s-3}+\frac{4!}{s^{5}}-5\frac{s}{s^2+1}$
\end{example}

\subsection{Inverse Laplace Transforms}

\begin{definition} 
	Given a function $f(t)$ who's Laplace Transform is $F(s)$, the inverse Laplace Transform is defined by $\laplace^{-1}[F(s)]=f(t)$
\end{definition}

This is a more algebraically involved procedure, as we need to transform cases to match the table.

\begin{example}

	\begin{enumerate}
		\item $\laplace^{-1}[\frac{12}{x}]=12$
		\item $\laplace^{-1}[\frac{1}{s-4}]=e^{4t}$
		\item $\laplace^{-1}[\frac{s}{s^2+16}]=\cos{4t}$
		\item $\laplace^{-1}[\frac{1}{s^2+16}]=?$\\
			This almost looks like case 6, $\frac{k}{s^2+k^2}$. What we can do is $\frac{1}{4}\laplace^{-1}[\frac{4}{s^2+16}]$ which becomes $0.25\sin{4t}$
		\item $\laplace^{-1}[\frac{s}{(s-2)^2+9}]$ resembles 7.
			\\Transform it to $\frac{s-2+2}{(s-2)^2+9}$, then $\frac{s-2}{(s-2)^2+9}+\frac{2}{(s-2)^2+9}$.\\
			Finally use table (transforming $\frac{2}{3}\frac{3}{(s-2)^2+9}$) to become $e^{2t}\cos{3t}+\frac{2}{3}e^{2t}\sin{3t}$
		\item $\laplace^{-1}[\frac{5}{s^2+3s+2}]$\\
			Factor: $\frac{5}{(s+2)(s+1)}$, decompose with partial fractions: $\frac{A}{s+2}+\frac{B}{s+1}, A(s+1)+B(s+2)=5.$\\
			$B(-1+2)=5, B=5, A(-2+1)=5, A=-5, \frac{-5}{s+2}+\frac{5}{s+1}$\\
			Now that its usable, use table: $-5e^{2t}+5e^{-t}$
	\end{enumerate}
\end{example}

\subsection{Initial value problems with Laplace Transforms}

If $y$ is a function of $t$, $\laplace[y(t)]=Y(s)$. What is the Laplace transform for its derivative, $y'$? From the definition, $\int_0^\infty e^{-st}y'(t)dt$, we can do 
integration by parts to get $-y(0)+s\laplace[y(t)]$, \begin{equation}\laplace[y']=sY(s)-y(0)\end{equation} (case 14).

Therefore, the steps involved are:
\begin{enumerate}
	\item Take the Laplace on both sides, and apply the initial condition right away.
	\item Solve algebraically for $Y(s)$.
	\item Find the inverse Laplace transform $y(t)$ of $\laplace^{-1}[Y(s)]$
\end{enumerate}

\begin{example}
	Solve using LT: $y'-3y=e^{5t}, y(0)=-1$\\
	Take Laplace of all terms: $\laplace[y']-3\laplace[y]=\laplace[e^{5t}], sY(s)-y(0)-3Y(s)=\frac{1}{s-5}$\\
	Plug in initial value: $sY(s)+1-3Y(s)=\frac{1}{s-5}$\\
	Algebra: $(s-3)Y(s)=\frac{1}{s-5}-1, Y(s)=\frac{1}{(s-5)(s-3)}-\frac{1}{s-3}$\\
	Find inverse of all terms: $y(t)=\laplace^{-1}[\frac{1}{(s-5)(s-3)}]-\laplace^{-1}[\frac{1}{s-3}]$\\
	Apply partial fractions to first: $1=A(s-3)+B(s-5), B=-0.5, A=0.5, \laplace^{-1}[\frac{0.5}{(s-5)}-\frac{0.5}{(s-3)}+[\frac{1}{s-3}]$\\
	All case 3, apply: $y(t)=0.5e^{5t}-1.5e^{3t}$
\end{example}

This itself is not too useful, but it gets easier and easier with higher order equations. For example, 2nd order equations have a Laplace transform $\laplace[y'']$
of \begin{equation} \laplace[y'']=s^2Y(s)-sy(0)-y'(0) \end{equation}


\begin{example} 
	Solve using LT: $y''+3y'+2y=0, y(0)=1, y'(0)=1$\\
	Take LT of all terms: $(s^2Y(s)-sy(0)-y'(0))+3(sY(s)-y(0))+2Y(s)=0, 
	\\s^2Y(s)-sy(0)-y'(0)+3Y(s)-3y(0)+2Y(s)=0$\\
	Plug in initial values: $s^2Y(s)-s-1+3sY(s)-3+2Y(s)=0, (s^2+3s+2)Y(s)-s-4=0$\\
	Solve for $Y(s)$: $Y(s)=(s+4)(s^2+3s+2)$ (Note, not a coincidence that denominator looks like characteristic eq.)\\
	Factor: $Y(s)=\frac{s+4}{(s+1)(s+2)}$\\
	Take Laplace (with partial frac): $s+4=A(s+2)+B(s+1), s=-2, 2=-B,B=-2, s=-1, A=1, A=3$.\\
	$Y(s)=\laplace^{-1}[\frac{3}{s+1}-\frac{2}{s+2}]$\\
	Find in chart: $Y(s)=3e^{-t}-2e^{-2t}$ (BONUS: We know the form is $C_1e^{rt}+C_2e^{ut}$ from previous chapters, so we know our answer here is correct as it matches!)
\end{example}

In general, a 2nd order equation $ay''+by'+cy=f(t)$ becomes $a(s^2Y(s)-sy(0)-y'(0))+b(sY(s)-y(0))+cY(s)=\laplace[f(t)]$.\\
Rewritten, it becomes $Y(s)=\frac{\laplace[f(t)]+asy(0)+ay'(0)+by(0)}{as^2+bs+c}$, at which point you apply partial fractions if needed.

\subsection{Shifting Theorems and Discontinuous Inputs}


\begin{theorem}[First Shifting Theorem]
	Let $\laplace[f(t)]=F(t)$.

	Then, $\laplace[e^{at}f(t)]=F(s-a)$
\end{theorem}

This is visible in case 2 in the table, $\laplace{t^n}=\frac{n!}{s^{n+1}}$, and case 4, $\laplace{e^{at}t^n}=\frac{n!}{(s-a)^{n+1}}$.
Case 5 and 7 share the same properties ($\cos{kt}$ and $e^{at}\cos{kt}$), and 6 and 8.

\begin{definition} [Unit Step Function]
	
	Defined as \\$U(t-a)$ or \\$H(t-a)=\begin{cases}
		0 & \text{if $0\leq t<a$}\\
		1 & \text{if $a \leq t$}

	\end{cases}$
\end{definition}

The step function is very useful for cases where some input can be toggled. We can use this to analyze differential equations with a toggle. 
These toggles can be added together as well, for example as $5U(t-2)-9U(t-8)$, with two switches toggled at the given times. A function like this can be 
written as $\begin{cases} 0 & 0 \leq t < 2 \\ 5 & 2 \leq t < 8 \\ -4 & 8 \leq t \end{cases}$.

In a generic case where the function toggles from a value of $f_1(t)$ to $f_2(t)$, it can be rewritten as $f(t)=f_1(t)+U(t-t_1)(f_2(t)-f_1(t))$. 
In general, a step function witn $n$ terms and $n-1$ different times can be written as $f_1+U(t-t_1)(f_2-f_1)+U(t-t_2)(f_3-f_2)+U(t-t_3)(f_4-f_3)...$

\subsubsection{Laplace transform of Heaviside function}

What is $\laplace[U(t-a)]$? Breaking up the integral into the terms from $0$ to $t_1$ and from $t_1$ to $\infty$, it becomes $\frac{e^{-as}}{s}$.


\begin{theorem}[Second Shifting Theorem]
	Let $\laplace[f(t)]=F(s)$.
	
	Then $\laplace[u(t-a)*f(t-a)]=e^{as}F(s)$
\end{theorem}

\begin{example} $\laplace[U(t-2)e^{3(t-2)}]$

	We see that the two $t-2$ terms match. We get $e^{-2s}$, and the func here is $f(t)=e^{3t}$, so answer is $\frac{e^{-2s}}{s-3}$.
\end{example}

\begin{example} $\laplace[U(t-7)\cos{4(t-7)}]$

	Again, $t-7$ match and $f(t)=\cos{4t}$ so we get $\frac{se^{-7t}}{s^2+16}$.
\end{example}

\begin{example} $\laplace[U(t-6)e^{2t}]$

	Harder here, we don't have $t-6$ in the exponent. If we force a $t-6$ in, we can add 12 to compensate and get $e^{2(t-6)}e^{12}$. We then get 
	\\$\laplace[U(t-6)e^{2(t-6)}e^{12}]=e^{12}\laplace[U(t-6)e^{2(t-6)}]=\frac{e^{12}e^{-6s}}{s-2}$
\end{example}

We can use this to create situations with toggling forces. For example, a 
spring mass system with $m=1,k=4$ and an external force of 4N at $t=3$
can be modeled as $y''+4y=4U(t-3)$.

\subsubsection{Inverse of second shifting theorem}

The operation is simple, $$\laplace^{-1}[e^{as}F(s)]=U(t-a)f(t-a)$$

\begin{example}
	$\laplace^{-1}[e^{-2s}\frac{3}{s^2-9}]=U(t-2)\sin{3(t-2)}$
\end{example}

\section{Chapter 4: Systems of Linear Equations}

\subsection{Basics of Matrices}
Given one linear equation (ex. $2x+3y=5$), how many x/y solutions are there? Since there's only one equation, there are infinite possible solutions. 
This can STILL BE IMPORTANT, but if you need one solution then this does not suffice.
Given 2 equations, how many possible solutions are there (ex. $x+y=5, 3x+4y=17$)? This one would have one solution, namely 3,2. There is, however, another 
situation with 2 solutions (ex. $x+y=5, 3x+3y=17$). This gives no possible solutions, and is known as an inconsistent system.

These linear equations can be stored in the form $a_{11}x_1+a_{12}x_2...a_{1n}x_n=b_1, a_{21}x_1+a_{22}x_2...a_{2n}x_n=b_2, ..., a_{m1}x_1+a_{m2}x_2...a_{mn}x_n=b_m$. 
This is a very unwieldy notation, but we can use matrices to represent these linear equations. We can insert the values as the rows/columns of a matrix, like so:
$$
\begin{bmatrix}
	a_{11} & a_{12} & a_{13} & ... & a_{1n} \\
	a_{21} & a_{22} & a_{23} & ... & a_{2n} \\
	a_{31} & a_{32} & a_{33} & ... & a_{3n} \\
	... & ... & ... & ... & ... \\
	a_{m1} & a_{m2} & a_{m3} & ... & a_{mn} \\
\end{bmatrix}
\begin{bmatrix}x_1 \\ x_2 \\ x_3 \\ .. \\ x_m\end{bmatrix} = \begin{bmatrix} b_1 \\ b_2 \\ b_3 \\ ... \\ b_m \end{bmatrix}
$$

This represents $A=m\times n$, using a matrix with $m$ rows and $n$ columns for the scalars and two $m$ length vectors, one for the unknowns and one for the
constants. This notation makes these notably easier to work with when it comes to calculations (especially with software). 

\subsubsection{Matrix operations}
Two matrices A and B can only be subtracted $\iff$ they have the same dimensions. The operation is simple, just add/subtract the corresponding entries of A and B.
\begin{example} $\begin{bmatrix} 1 & 2 \\ 3 & 4 \end{bmatrix} + \begin{bmatrix} 1 & 0 \\ 2 & 1 \end{bmatrix} = \begin{bmatrix} 2 & 2 \\ 5 & 5 \end{bmatrix}$
\end{example}

Multiplication is not as simple. First of all, matrices A and B can be multiplied to AB $\iff n_a=m_b$. The amount of rows of A or columns of B do not matter.
The resulting matrix is a matrix of size $m_a\times n_b$. This means that matrix multiplication is NOT commutative. The operation itself takes each row of A
and multiplies it with each column of B. For example,
$$\begin{bmatrix}a_{11} & a_{12} \\ a_{21} & a_{22} \end{bmatrix} * \begin{bmatrix}b_{11} & b_{12} & b_{13} \\ b_{21} & b_{22} & b_{23}\end{bmatrix}=$$
$$
\begin{bmatrix}
a_{11}b_{11}+a_{12}+b_{21} & a_{11}b_{11}+a_{12}+b_{22} & a_{11}b_{13}+a_{12}+b_{23} \\ 
a_{21}b_{11}+a_{22}+b_{21} & a_{21}b_{12}+a_{22}+b_{22} & a_{21}b_{13}+a_{22}+b_{23}
\end{bmatrix}$$

\begin{example} $\begin{bmatrix} 1 & 2 \\ 3 & 4 \\ 5 & 6 \end{bmatrix} * \begin{bmatrix} 2 \\ 5 \end{bmatrix}$
	
	Since sizes are $3\times 2, 2\times 1$, they result in a matrix of size $3\times 1$
	$$AB=\begin{bmatrix} 2+10 \\ 6 + 10 \\ 10 + 30\end{bmatrix}=\begin{bmatrix}12\\26\\70\end{bmatrix}$$
\end{example}

\begin{example} $\begin{bmatrix} 1&2\\3&4\\0&4\end{bmatrix} * \begin{bmatrix}1&0\\0&1\end{bmatrix}$
	$$AB=\begin{bmatrix} 1*1+2*0 & 1*0+2*1 \\ 3*1+0*4 & 3*0+1*4 \\ 0*1+0*4 & 0*0+1*4\end{bmatrix} = 
	\begin{bmatrix} 1 & 2 \\ 3 & 4 \\ 0 & 4 \end{bmatrix}$$
\end{example}

In the above example, we saw how the matrix $\begin{bmatrix}1&0\\0&1\end{bmatrix} $ gave the same value after multiplication. This is an example of the 
identity matrix, in this case for matrices of 2 columns. In general, the identity matrix follows the given pattern:
$$\begin{bmatrix}1&0&0&...&0\\0&1&0&...&0\\0&0&1&...&0\\...&...&...&...&...\\0&0&0&...&1\end{bmatrix}$$

Importantly, given a matrix A, $AI=IA=A$.

Other special matrices are:
\begin{itemize}
	\item A square matrix is a matrix where $m=n$.
	\item An upper triangular matrix is a square matrix where all the elements below the main diagonal are 0.
	\item A lower triangular matrix is a square matrix where all elements above the main diagonal are 0.
	\item A diagonal matrix is a square matrix where all the elements outside of the main diagonal are 0 (both lower and upper triangular).
	\item As seen before, an identity matrix is a diagonal matrix with all diagonal elements being 1.
\end{itemize}

\end{document}
