\documentclass{article}
\usepackage[utf8]{inputenc}
\title{Physics 2 for Engineers: PHYS1155}
\author{Javier Coindreau}

\newtheorem{definition}{Definition}
\newtheorem{theory}{Theory}
\newtheorem{example}{Example}
\newtheorem{derivation}{Derivation}

\usepackage{amssymb}
\usepackage{enumerate}

\begin{document}


\begin{titlepage}
	\maketitle
	\tableofcontents{}
\end{titlepage}
\section{Electric Potential}
\subsection{Electric Potential Energy}
\subsubsection{Movable charges in a field}

What happens to charges that are free to move? Do they accelerate at a constant rate?

We could use the known forces of electric fields to determine their acceleration over time, but that results in multiple differentials due to the changing distance $r$ as charges move. It results in two coupled differential equations that both depend on $r_1,r_2$.

We can, however, move to viewing forces through terms of energy to determine changes in state between points.

\begin{definition}
	A conservative force is a force whose work done depends only on the initial and final positions, and for which we can define a potential energy function $U$
	\begin{equation} U(x)=-\int{F \cdot d\vec{x}} \end{equation}
\end{definition}

From this, we can use electric potential to translate electric forces to mechanical work.

This works because electricity is a conservative force, as only movement along electric field lines do any work.

For electrostatics, the potential is defined as 
\begin{equation} U_e=-\int{\frac{q_1 q_2}{4\pi\epsilon_0 r^2} r\hat \cdot dr}, U_e=\frac{q_1 q_2}{4\pi\epsilon_0 r}=\frac{kq_1 q_2}{r}
\end{equation}

Can be positive or negative, and units are joules.

\begin{example}
	Two protons are separated by 1cm and not in motion relative to one another. They are not fixed in place.
	\begin{enumerate}[a)]
		\item What is the initial potential energy of the configuration?

			Plug in: $ \frac{k e^2}{r_1}=\frac{8.99*10^9*1.6*10^{-19}}{0.01}=2.3*10^{-26} J$
		\item How fast are they moving at time $t$?

			$E_f=E_j, \frac{ke^2}{r_1}+0=\frac{ke^2}{r_2}+0.5 m v^2$
			\\The potential energy here is used as the energy of the entire system. Because of that, we need to multiply the kinetic energy times 2 to become $\frac{ke^2}{r_1}+0=\frac{ke^2}{r_2}+m v^2$

			Solve:
			$v=\sqrt{(\frac{ke^2}{r_1}-\frac{ke^2}{r_2})\frac{1}{m}}=3.62 m/s$
		\item How much work is done to bring them back together?

			$W_{me}=-W_e=\Delta U_E=U_1-U_2=\frac{ke^2}{r_1}-\frac{ke_2}{r_2}=mv^2=2.18*10^{-26} J$
	\end{enumerate}
\end{example}
\subsubsection{Special case of uniform electric fields}
When electric fields are constant, we can simplify the potential energy to rely on the field itself.

\begin{equation} F=qE=constant \iff \Delta U_e= q E\Delta r \end{equation}

	$\Delta r$ is the distance through the electric field. Much like gravitational potential energy, you must set a reference point to solve problems involving constant fields. 


\end{document}
